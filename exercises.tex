\documentclass[10pt,english,a4paper]{article}
\usepackage[T1]{url}
\usepackage[utf8]{inputenc}
\usepackage{graphicx, varioref, babel, fancyvrb, listings, amsmath,amsthm, amssymb, enumerate}
\usepackage{listings}
\usepackage{mathtools}
\usepackage{csquotes}
\usepackage{tikz}
\usepackage{tikz-cd}
\usepackage{xcolor}

% mathbf vectors
\def\a{\boldsymbol{a}}
\def\b{\boldsymbol{b}}
\def\c{\boldsymbol{c}}
\def\dv{\boldsymbol{d}} % OBS
\def\e{\boldsymbol{e}}
\def\f{\boldsymbol{f}}
\def\g{\boldsymbol{g}}
\def\h{\boldsymbol{h}}
\def\i{\boldsymbol{i}}
\def\j{\boldsymbol{j}}
\def\k{\boldsymbol{k}}
\def\l{\boldsymbol{l}}
\def\m{\boldsymbol{m}}
\def\n{\boldsymbol{n}}
\def\ov{\boldsymbol{o}} % OBS -> compilation error with 'ø'
\def\p{\boldsymbol{p}}
\def\q{\boldsymbol{q}}
\def\rv{\boldsymbol{r}} % OBS
\def\s{\boldsymbol{s}}
\def\t{\boldsymbol{t}}
\def\u{\boldsymbol{u}}
\def\v{\boldsymbol{v}}
\def\w{\boldsymbol{w}}
\def\x{\boldsymbol{x}}
\def\y{\boldsymbol{y}}
\def\z{\boldsymbol{z}}

% Matrices
\def\A{\mathbf{A}}
\def\B{\mathbf{B}}
\def\Cm{\mathbf{C}} % OBS
\def\D{\mathbf{D}}
\def\E{\mathbf{E}}
\def\F{\mathbf{F}}
\def\G{\mathbf{G}}
\def\H{\mathbf{H}}
\def\I{\mathbf{I}}
\def\J{\mathbf{J}}
\def\K{\mathbf{K}}
\def\L{\mathbf{L}}
\def\M{\mathbf{M}}
\def\Nm{\mathbf{N}} % OBS
\def\Ov{\mathbf{O}} % OBS -> compilation error with 'Ø'
\def\P{\mathbf{P}}
\def\Qm{\mathbf{Q}} % OBS
\def\Rm{\mathbf{R}} % OBS
\def\S{\mathbf{S}}
\def\T{\mathbf{T}}
\def\U{\mathbf{U}}
\def\V{\mathbf{V}}
\def\W{\mathbf{W}}
\def\X{\mathbf{X}}
\def\Y{\mathbf{Y}}
\def\Zm{\mathbf{Z}} % OBS

% Script letters
\newcommand{\As}{\mathcal{A}}
\newcommand{\Bs}{\mathcal{B}}
\newcommand{\Cs}{\mathcal{C}}
\newcommand{\Ds}{\mathcal{D}}
\newcommand{\Es}{\mathcal{E}}
\newcommand{\Fs}{\mathcal{F}}
\newcommand{\Gs}{\mathcal{G}}
\newcommand{\Hs}{\mathcal{H}}
\newcommand{\Is}{\mathcal{I}}
\newcommand{\Js}{\mathcal{J}}
\newcommand{\Ks}{\mathcal{K}}
\newcommand{\Ls}{\mathcal{L}}
\newcommand{\Ms}{\mathcal{M}}
\newcommand{\Ns}{\mathcal{N}}
\newcommand{\Os}{\mathcal{O}}
\newcommand{\Ps}{\mathcal{P}}
\newcommand{\Qs}{\mathcal{Q}}
\newcommand{\Rs}{\mathcal{R}}
\newcommand{\Ss}{\mathcal{S}}
\newcommand{\Ts}{\mathcal{T}}
\newcommand{\Us}{\mathcal{U}}
\newcommand{\Vs}{\mathcal{V}}
\newcommand{\Ws}{\mathcal{W}}
\newcommand{\Xs}{\mathcal{X}}
\newcommand{\Ys}{\mathcal{Y}}
\newcommand{\Zs}{\mathcal{Z}}

% Often used number systems. 
\def\C{\mathbb{C}}
\def\Z{\mathbb{Z}}
\def\N{\mathbb{N}}
\def\Q{\mathbb{Q}}
\def\R{\mathbb{R}}

% dx
\def\d{~\text{d}}
\def\dla{~\text{d}\lambda}
\def\dmu{~\text{d}\mu}
\def\dnu{~\text{d}\nu}

% Words
\def\dim{\text{ dim}}
\def\conv{\textnormal{conv}} 
\def\card{\textnormal{card}}
\def\supp{\textnormal{supp}}
\def\rank{\textnormal{rank}}

% Arrows
\def\Uda{\Updownarrow}
\def\ua{\uparrow}
\def\ra{\rightarrow}
\def\Ra{\Rightarrow}
\def\Lra{\Longleftrightarrow}
\def\lra{\leftrightarrow}
\def\laa{\leftarrow}
\def\Laa{\Leftarrow}
\def\Da{\Downarrow}

% dots
\def\ld{\ldots}
\def\vd{\vdots}
\def\cd{\cdots}
\def\dd{\ddots}

% Braces
\def\({\left(}
\def\){\right)}



%Greek Letters
\def\eps{\epsilon}
\def\la{\lambda}
\def\La{\Lambda}
\def\om{\omega}
\def\Om{\Omega}
\def\al{\alpha}
\def\Al{\Alpha}
\def\ga{\gamma}
\def\Ga{\Gamma}

% others
\def\pa{\partial}
\def\ul{\underline}
\def\ol{\overline}
\def\ob{\overbrace}
\def\ub{\underbrace}
\def\bs{\boldsymbol}
\def\ti{\times}



%newcommand
\newcommand{\ind}[1]{\left\langle #1 \right\rangle}
\newcommand{\indi}[1]{\langle #1 \rangle}
\newcommand{\floor}[1]{\left\lfloor #1 \right\rfloor}
\newcommand{\ceil}[1]{\left\lceil #1 \right\rceil}
\newcommand{\nlist}[1][]{{#1}_1, {#1}_2, \ldots, {#1}_{n}}
\newcommand{\dlist}[2][]{{#1}_1, {#1}_2, \ldots, {#1}_{#2}}
\newcommand{\oo}[1][2]{\frac{1}{#1}}
\newcommand{\po}[1][2]{\frac{\pi}{#1}}
\newcommand{\Span}[1]{\text{span} \left\{#1\right\} }
\newcommand{\File}[1]{\fvset{fontsize=\small, frame=lines}%,
                            % numbers=left, numbersep=3pt} 
                            \VerbatimInput{#1}}


%\declaretheorem[style = remark, sibling = lemma, qed = $\maltese$, name = "]{thm}

\makeatletter
\def\moverlay{\mathpalette\mov@rlay}
\def\mov@rlay#1#2{\leavevmode\vtop{%
   \baselineskip\z@skip \lineskiplimit-\maxdimen
   \ialign{\hfil$\m@th#1##$\hfil\cr#2\crcr}}}
\newcommand{\charfusion}[3][\mathord]{
    #1{\ifx#1\mathop\vphantom{#2}\fi
        \mathpalette\mov@rlay{#2\cr#3}
      }
    \ifx#1\mathop\expandafter\displaylimits\fi}
\makeatother

\newcommand{\cupdot}{\charfusion[\mathbin]{\cup}{\cdot}}
\newcommand{\bigcupdot}{\charfusion[\mathop]{\bigcup}{\cdot}}



\theoremstyle{definition}
\newtheorem*{definition}{Definition}
\newtheorem*{theorem}{Theorem}
\newtheorem*{lemma}{Lemma}
\newtheorem*{corollary}{Corollary}
\newtheorem*{proposition}{Proposition}
\newtheorem*{observation}{Observasjon}
\newtheorem*{example}{Example}
\newtheorem*{note}{Note}
\newtheorem*{remark}{Remark}

\let\emph\relax % there's no \RedeclareTextFontCommand
\DeclareTextFontCommand{\emph}{\bfseries}
%\begin{figure}
%    \centering
%    \includegraphics[width=0.8\linewidth]{nameOfFile}
%    \caption{}
%\end{figure}

\def\Psfin{\mathcal{P}_{\textnormal{fin}}}
\def\Fbb{\mathbb{F}}
\def\Tbb{\mathbb{T}}
\def\Ff{\mathfrak{F}}
\def\Kf{\mathfrak{K}}
\def\smin{\setminus}
\def\ex{\textnormal{ex}\,}
\def\co{\textnormal{co}\,}
\def\whAs{\widehat{\As}}
\def\tAs{\widetilde{\As}}
\def\Assa{\As_{\text{sa}}}
\DeclareMathOperator{\Id}{Id}
\DeclareMathOperator{\Tr}{Tr}
\DeclareMathOperator{\Sp}{sp}
\def\vphi{\varphi}
\def\top{\tilde{\oplus}}
\newcommand{\wt}[1]{\widetilde{ #1 }}
%%% Colours %%%
\newcommand*{\BLUE}[1]{{\color{blue}#1}}
\newcommand*{\RED}[1]{{\color{red}#1}}
\newcommand*{\GREEN}[1]{{\color{green}#1}}


\title{Exercises MAT4360}
%\author{Vegard Antun \\MAT 4450 }

\begin{document}
\maketitle

\subsection*{Ex 1}
\begin{enumerate}[a)]
    \item 
Assume $T \colon H \to H$ is a linear operator which is weak-weak continuous. Then
$T \in \Bs(H)$.
\begin{proof}
    Assume $x_n \to x$ in norm in $H$ and $Tx_n \to y$  in $H$. Then 
    $x_n \to x$ weakly and $Tx_n \to y$ weakly. From the assumption we also get that 
    $Tx_n \to Tx$ weakly. Since ($H_1$, weak top.) is Hausdorff we get that 
    $Tx = y$. Hence the graph of $T$ is closed, so $T \in \Bs(H)$ by the closed 
    graph theorem.
\end{proof}

\item Then $T \in \Bs(H)$ if and only if $T$ is norm-weak continuous.
\begin{proof}
    $\implies$ Follows immediately from the fact that norm-convergence is stronger than weak convergence.

    $\impliedby$ Follows from the closed graph theorem along similar lines as those above. 
\end{proof}

\end{enumerate}

\subsection*{Ex 2}
$\Ks(H)$ is unital if and only if $H$ is finite dimensional. 

\begin{proof}
    If $H$ is finite dimensional, then $\Ks(H)=\Bs(H)$ and $I_{H} \in \Bs(H)$.

    Conversely assume $\Ks(H)$ is unital, with unit $E$. Let $\xi \in H_1$. Let 
    $P_{\xi}$ be the orthogonal projection from $H$ onto $\C \cdot \xi$. Then
    $P_{\xi} \in \Ks(H)$. Moreover $EP_{\xi} = P_{\xi}$, hence $(\ub{EP_{\xi}}_{E(\xi)})(\xi) = \ub{P_{\xi}\xi}_{\xi}$.
    It follows that $E = I_{H}$, so $I_H \in \Ks(H)$, which implies that $H$ is finite
    dimensional (otherwise, we can pick an orthonormal sequence $\{\xi_n\}$ in $H$),
    but $I_H(\xi_n) = \xi_n$ has a non-convergent subsequence, since 
    $||\xi_n - \xi_m||^2 = 2$ for all $n \neq m$.
\end{proof}


\section*{Ex 4}
$H$ is Hilbert space. $T\colon H \to H$ linear operator. Show that 
$T \in \Fs(H)$ if and only if $T$ is weak-norm continuous. 

\begin{proof}
    $\implies$ Recall that we can write any $T \in \Fs(H)$ as a finite sum of 
    operators of the form 
    \[ x \overset{T_{\eta,\xi}}{\to} \ind{x,\eta}\xi, \quad \text{where }\eta,\xi \in H\]
   It is obvious that all these rank one operators is weak-norm continuous.
    (Assume $x_i \to x$ weakly in $H$. Then 
    \[||T_{\eta,\xi}(x_i) - T_{\eta,\xi}(x)|| = ||\ind{x_i-x,\eta}\xi|| 
    = |\ind{x_i-x,\eta}|||\xi|| \to 0\]
which converges weakly by def.)  So $T$ is also weak-norm continuous.


$\impliedby$ Assume $T$ is weak-norm continuous. Hence $T \in \Bs(H)$. Set 
$U = \{ x \in H : ||x|| < 1 \}$. Then $T^{-1}(U)$ is weakly open in $H$
and contains $0$. This means that we can find $\eta_1,\ldots, \eta_n \in H$ 
and $\eps > 0$ such that 
\[ V \coloneqq \{ x \in H : \ind{x,\eta_i} < \eps, i =1, \ldots, n \}  \subseteq T^{-1}(U)\]
i.e. we have $||Tx|| < 1$ whenever $x \in V$. Let now $P$ denote the orthogonal 
projection onto the span of $H$ onto $M = \Span{\eta_1, \ldots,\eta_n}$. 
Note that $y \in M^{\perp} \implies \ind{y,\eta_i} = 0$, $i=1,\ldots, n$,
which implies that $y \in V$. $(I-P)x \in V$ for all $x \in H$. So we get that
$||T(I-P)x|| < 1$ for all $x \in H$. It follows that 
$||T(I-P)\xi|| = \tfrac{1}{m}||T(I-P)m\xi||< \tfrac{1}{m}$ for all $\xi \in H$,
$m \in \N$. This implies that 
\begin{align*}
    T(I-P)\xi &= 0&&\forall \xi \in H \\  
    T(I-P) &= 0 \\
    T-TP &= 0
\end{align*}
Hence $T = TP$ which has finite range, since the range of $P$ is finite dim. i.e.
$T \in \Fs(H)$.


\end{proof}



\section*{Ex 5}
Let $H$ be an infinite dimensional Hilbert space. Then $\Ks(H) + \C I_{H}$ is a 
$C^*$-subalgebra of $\Bs(H)$ which is isometricly $*$-isomorphic to
$\widetilde{\Ks(H)}$. 

Indeed, define $\phi\colon \widetilde{K(H)} \to \Ks(H) + \C I_{H} \subseteq \Bs(H)$,
by $\phi((K,\la)) = K + \la I_{H}$ for $K  \in \Ks(H)$ and $\la \in \C$.
It is easy to see that $\phi$ is a $*$-homomorphism as a map from 
$\wt{\Ks(H)}$ into $\Bs(H)$, so its range $\Ks(H) + \C I_{H}$ is a $*$-subalgebra
of $\Bs(H)$, which is norm closed, since $K(H)$ is norm closed and $\C I_{H}$
is a finite dimensional subspace. Hence $\Ks(H) + \C I_H$ is a $C^*$-subalgebra
of $\Bs(H)$.

Note that $\phi\colon \tilde{K(H)} \to \Ks(H) + \C  I_H$ is a $*$-isomorphism. 
surjectivity is automatic. We only have to check that it is injective. 
Assume $\phi((K,\la)) \to 0$ i.e. $K + \la I_{H} = 0$, so $K = -\la I_H$.
If $\la \neq 0$ we have that $I_H = -\tfrac{1}{\la}K  \in \Ks(H)$ which is impossible,
since $H$ is infinite dimensional. So we must have $\la = 0$, but then $K=0$. 


\section*{Ex 6}
$\Om$ locally compact Hausdorff space. 
\begin{align*}
    C_{0}(\Om) = \{ f \in C_b(\Om) :\forall \eps >0, \exists K \subseteq \Om, K \text{ compact, }\text{s.t. }|f(\om)| < \eps \forall \om \in \Om \setminus K   \}
\end{align*}
is a commutative $C^*$-algebra. 

\begin{enumerate}[(a)]
    \item $C_0(\Om)$ is unital if and only if $\Om$ is compact

    \begin{proof}
        If $\Om$ is compact, then $C_0 (\Om) = C(\Om)$ and $1_{\Om} \in C_0(\Om)$.

        On the other hand, assume $C_0(\Om)$ has a unit, say $e$. For each $\om \in \Om$,
        pick $f_{\om} \in C_0 (\Om)$ such that $f_{\om}(\om) = 1$.
   
        Idea: $\tilde{\Om} = \Om \cup \{\infty\}$ is a $1$-point compacification. We can identify
        $C_0(\Om)$ with $\{ g \in C(\tilde{\Om}) : g(\infty) =0 \}$ by $f \in C_{0}(\Om)$
        and the map $f \mapsto \tilde{f}$ by 
    \[\tilde{f} (x) \begin{cases} f(x) & x \in \Om \\ 0 & x =\infty \end{cases} \]
    So given $\om \in \Om$ we can find $g_{\om} \in C(\tilde{\Om})$ such that 
    $g_{\om}(\om)= 1$ and $g_{\om}(\infty) = 0$, using Urysohn's lemma.
    
    Then we have $e(\om) = (ef_{\om})(\om) = f_{\om}(\om)= 1$, so $e(\om)=1$ 
    for all $\om \in \Om$. Hence $e = 1_{\Om}$ so $1_{\Om} \in C_0(\Om)$
    which is possible only if $\Om$ is compact. 
\end{proof}


\item 
    Assume now that $\Om$ is not compact. Set $\tilde{\Om} = \Om\cup\{\infty\}$.
    We want to show that $\wt{C_0(\Om)}$ is isometrically $*$-isometric to
    $C_0(\tilde{\Om})$. 



\begin{proof}
    Define $\phi\colon \tilde{C_0(\Om)} \to C(\tilde{\Om})$ by 
    $\phi((f,\la)) = \tilde{f} + \la 1_{\tilde{\Om}}$ where 
    $\tilde{f}$ is defined as above. It is elementary to check that 
    $\phi$ is a $*$-homomorphism.\ul{$\phi$ is injective:}
    If $\phi(f-\la) = 0$ then $\tilde{f} = -\la 1_{\tilde{\Om}}$. If 
    $\la \neq 0$ we have $1_{\tilde{\Om}} = -\frac{1}{\la} \tilde{f}$, so 
    $1_{\Om} = \frac{1}{\la}f$
    which implies that $\Om$ is compact, which is impossible. If $\la =0$, 
    we have $\tilde{f} =0$, so $f = 0$, so we get inactivity.

    \ul{$\phi$ is surjective:} Let $g \in C(\tilde{\Om})$. Set $\la = g(\infty)$ 
     and define $f \in C_0(\Om)$ by $f(\om) = g(\om)-\la$ for $\om \in \Om$.
    Then we get that $\phi(f,\la) = g$. 

    This means that $\phi$ is a $*$-isomorphism as desired. 


\end{proof}


\end{enumerate}

\section*{Ex 7}
$\As$ unital $C^*$-algebra. $A \in \Assa$. $f,g \in C(\Sp(A))$ real valued functions,
so $f(A)$, $g(a) \in \Assa$.

\begin{enumerate}[(a)]
    \item 
$f(A) \leq f(B)$ if and only if $f \leq g$.
Indeed, 
\begin{align*}
    f(A) \leq g(A) &\iff (g-f)(A)\geq 0 \\
    &\iff \Sp((g-f)(A)) \subseteq [0,\infty) \\
    &\iff (g-f)(\Sp(A)) \subseteq [0,\infty) \\
    &\iff g-f \geq 0 \quad \text{on }\Sp(A) \\
    &\iff f \leq g
\end{align*}
Since $\la \leq ||A||$ for all $\la \in \Sp(A)$ we have $\Id \leq ||A||1$ on $\Sp(A)$,
so we have 
$\Id(A) \leq ||A|| 1(A) $ which implies $A \leq ||A|| I$ and similarly for
$-||A|| I \leq A$.

\item Let $X \in \As$. Then $X^*X \leq I$ if and only of $||X||\leq 1$. Set 
$A = X^*X \in \As^+ \subseteq \Assa$. Note that 
$X^*X \leq I ~\iff ~ A \leq I ~ \iff ~ \Id \leq 1$ on $\Sp(A)$.
\begin{align*}
    &\iff ||A|| \leq 1 \\
    &\iff ||X||^2 \leq 1 \\
    &\iff ||X|| \leq 1
\end{align*}
Note: $\Sp(A) \subseteq [0,\infty)$ and $||A|| \in \Sp(A)$, because 
$||A|| = \sup\{ \la \in \Sp(A) \} \in \ol{\Sp(A)} = \Sp(A)$.

\end{enumerate}

\section*{Ex 8}
$A,B \in \As^+$. Then $A \leq B \implies ||A||\leq ||B||$.
\begin{proof}
    We can assume that $\As$ is unital, so suppose $A \leq B$.
    From ex 7a we know $B \leq ||B|| I$. Hence $A\leq ||B||I$ so 
    $\Id \leq ||B|| 1$ on $\Sp(A)$. Since $||A|| \in \Sp(A)$, wet get $||A||
    \leq ||B||$. 
\end{proof}



\section*{Ex 9}
$\As$ $C^*$-algebra, $A, B \in \As$. Show that $\Sp(AB) \cup \{0\} =
\Sp(BA)\cup \{0\}$.
\begin{proof}
    We may assume that $\As$ is a unital.
    By symmetry it suffices to show that one of the inclusions holds. 
    Let $\la \in \C \setminus (\Sp(AB)\cup \{0\})$. 
    Then $\la I -AB \in GL(\As)$ and $\la \neq 0$, so
    $I-\ub{\tfrac{1}{\la}A}_{A'}B \in GL(\As)$.  Set $C = (I-A'B)^{-1}$, so
    \[C(I-A'B) = (I-A'B)C = I\]. Want to find $D \in \As$ such that 
    \[D(I-BA') = (I-BA')D = I\]
    Set $D = I+BCA$. Then we get 
    \begin{align*}
        D-DBA' &= I-BCA' -BA' -BCA'BA' \\
        &= I+BCA' - BA' -BCA' +BA' = I
    \end{align*}
    Similarly $D-BA'D = I$. This means that $I-BA' \in GL(\As)$, so
    $\la I - BA = \la(I-BA') \in GL(\As)$, $\la  \neq 0$.
    Hence $\la \in \Sp(BA)\cap \{0\}$. 
     
\end{proof}

\section*{Ex 10}
$A \in \As^+$. Show $AB \in \As^+ \iff AB \iff BA$. If $AB \in \As^+$, 
then $AB = (AB)^* = BA$, since $A,B$ are self adjoint. 

Conversely, assume that $AB = BA$. Then $(AB)^* = B^*A^* = BA =AB$,
without loss of generality, we may assume that $\As$ is unital. 
Let $\Bs$ be the $C^*$-subalgebra of $\As$ generated by $A,B$ and $I$.
Note that 
\[
\Bs = \ol{\{ P(A,B) : P \text{ is a complle polynomial in two commuting variabels}\}}^{||\cdot||}
\]
so $\Bs$ is commutative and unital, since $A$ and $B$ commute.
We get that 
$\Sp_{\As}(AB) = \Sp_{\Bs}(AB) = \{\varphi{AB} : \varphi \in \widehat{\Bs}\}
= \{ \varphi(A)\varphi(B) : \varphi \in \widehat{\Bs} \}$. 
But $\varphi(A) = \varphi(C^*C) = |\varphi(C)|^2 \geq 0$, where $A = C^*C$, 
and similarly $\varphi(B) \geq 0$ for all $\varphi \in \widehat{\Bs}$.
So we get t 
\[ \Sp_{\As}(AB) \subseteq [0,\infty) \]
as desired. 
Could also use 
\begin{align*}
   \{ \varphi(A)\varphi(B) :\varphi \in \widehat{\Bs} \} 
    &\subseteq \ub{\{ \varphi(A) :\varphi \in \widehat{\Bs} \}}_{\Sp_{\Bs}(A)} \cdot
    \ub{\{ \varphi(B) :\varphi \in \widehat{\Bs} \}}_{\Sp_{\Bs}(B)}  \\
\end{align*}
where $\Sp_{\Bs}(A) = \Sp_{\As}(A) \subset [0,\infty)$ and similarly for $B$.
Let $A,B \in \Assa$, $C \in \As$. Then $A \leq B~\implies ~ C^*AC \leq C^*BC$. 
Indeed, assume $B-A \geq 0$, so $B-A = D^*D$ for some $D \in \As$. 
Then $C^*BC - C^*AC = C^*(B-A)C = C^*D^*DC = (DC)^*DC \geq 0$.


\section*{Exercise 12}
\subsection*{a)}
Let $e_n \in C_0(\R)$ be 1 for $|x| \leq n$ and 0 on $|x|> n+1$
Show that $\{e_n\}$ is an approximate unit to $C_0(\R)$. Let $f \in C_0(\R)$ and let 
$\eps > 0$. Pick a compact $K \subseteq \R$ such that 
$|f(x)| < \eps$ for all $x \in \R\setminus K$.
Note that 
\[ (f - e_n) = \begin{cases} 0 & |x| < n\\
f(x)(1-e_n(x)) & n < |x|\leq n+1 \\
f(x) & |x|> n+1    
\ \end{cases}\]
Choose $N$ such that $K \subseteq [-N,N]$ for any $n \geq N$ we have 
$K \subseteq [-n,n]\subseteq \R\setminus K$,
so we get 
\begin{align*}
    ||f-fe_n||_{\infty} \leq \sup\{ |f(x)|: |x|> n \} \leq \eps.
\end{align*}
Hence $||f-fe_n||_{\infty}\to 0$.

\subsection*{b)}
$H$ is a Hilbert space. $\{e_j\}_{j \in J}$ orthonormal basis for $H$. 
$\Vs = \{ F \subseteq J: F \text{ nonempty and finite} \} $ ordered by inclusion. 
For $F \in \Vs$, 
\[P_F = \text{orthonormal projection from $H$ onto $\Span{e_j:j\in F}$}\]
\ul{$\{P_F\}$ is an approximative unit for $\Ks(H)$}: 
Choose $K \in \Ks(H)$, and $\eps> 0$. Choose 
$K_0 \in \Fs(H)$ such that $||K - K_0|| < \frac{\eps}{3}$. We have seen 
that we may write $K_0$ in the form 
$K_0(x) = \sum_{i=1}^{n} \ind{x,\eta_i}\xi_i$ for some $\xi_1,\ldots, \xi_n\in H$
and $\eta_1, \ldots, \eta_n \in H \setminus \{0\}$ for all $x \in H$.
So 
\begin{align*}
||(K_0 -P_F K_0)(x)|| &= ||\sum_{k=1}^{n}\ind{x, \eta_k}(\xi_k - P_F\xi_k) || \\
&\leq \sum_{k=1}^{n} |\ind{x,\eta_k}|\,||\xi_k - P_F \xi_k ||
&\leq \sum_{k=1}^{n} ||x||\,||\eta_k||\,||\xi_k - P_F \xi_k || \\
&\leq \( \sum_{k=1}^{n}  ||\eta_k||\,||\xi_k - P_F \xi_k || \)||x||
\end{align*}
for all $x\in H$.

Now for $\xi \in H$, we have 
\[ P_F \xi = \sum_{j \in F} \ind{\xi,e_j}e_j \to \xi \]
i.e. $||\xi - P_F\xi|| \to 0$, so we see that we can find 
$F_0 \in \Vs$ such that 
$||\xi_k - P_F\xi_k|| < \frac{\eps}{3n||\eta_k||}$
for all $F \in \Vs$, with $F_0 \subseteq F$, $k = 1,\ldots, n$.
For any $F \in \Vs$, $F_0 \subseteq F$ we get 
\begin{align*}
    ||(K_0 - P_{F}K_0)(x)|| \leq \( \sum_{k=1}^{n} ||\eta_k|| \frac{\eps}{2n||\eta_k||} \)||x||
    = \frac{\eps}{3} ||x||
\end{align*}
i.e. $||K_0- P_F K_0|| < \frac{\eps}{3}$. All together, for $F$ as above, we get 
\begin{align*}
    ||K-P_FK|| &\leq ||K-K_0|| + ||K_0 - P_FK_0|| + ||P_F K_0 -P_F K|| \\
    &\leq ||K-K_0|| + ||K_0 - P_FK_0|| +||P_F|| \,||K_0-K|| \\
    &\leq 2||K_0-K|| + ||K_0 - P_FK_0|| < \eps 
\end{align*}
so $||K-P_FK|| \to 0$. Using that $\Ks(H)$ is self-adjoint we also get 
$||K-KP_F|| \to 0$. 
\BLUE{(Notice that $P_F$ does only converge in the strong operator topology, not in norms.)? }

\section*{Exercise 13}
\subsection*{a)}
$\Cs$ is a $*$-algebra. $\As,\Bs$ are $*$-subalgebras. 
Assume $\Cs = \As + \Bs$, $\As\cap\Bs = \{0\}$ and $\As\Bs = \{0\}$.
Then $\As,\Bs$ are two sided ideal in $\Cs$ and $\Cs$ is $*$-isomorphic to $\As \oplus \Bs$.
\begin{proof}
Consider $A \in \As$, and $C \in \Cs$. Write $C = A' + B'$, with $A' \in \As$, $B' \in \Bs$.
Then $CA = (A'+B')A = A'A + BA = A'A$. Here we used the fact that $BA \in \As\Bs = \{0\}$, 
since $\As\Bs = 0$ implies that $\Bs^*\As^* = \Bs\As= \{0\}$.
Similarly $AC \in \As$. In the same way $\Bs$ is an ideal in $\Cs$.    

\underline{Moreover $\Cs$ is $*$-isomorphic to $\As \oplus \Bs$:}
Define $\pi\colon \As\oplus\Bs \to \Cs$ by 
$\pi((A,B)) = A+ B$. Then 
$\pi$ is a $*$-isomorphism. Indeed, $\pi$ is obviously surjective and injective 
because if $\pi((A,B)) = \pi((A',B'))$ then 
$A+B = A'+B'$, so $A-A' = B'-B$, where $A-A' \in \As$ and $B'-B\in \Bs$, so
$A=A'$ and $B=B'$, since $\As \cap \Bs = \{0\}$.

$\pi$ preserves product:
\begin{align*}
    \pi((A,B)(A',B')) &= AA' + BB' \\
    \pi((A,B))\pi((A',B')) &= (A+B)(A'+B') = AA' + BA' + AB' + BB' \\
                           &= AA' + BB'  
\end{align*}
similarly one check that $\pi$ is linear and preserves $*$-operation. 
\end{proof}

\subsection*{b)}
Assume now $\Cs$ is a $C^*$-algebra and $\As,\Bs$ are $C^*$-subalgebras 
of $\Cs$, satisfying $\Cs = \As + \Bs$, $\As\Bs = \{0\}$. 
Note that \ul{we have $\As \cap \Bs =  \{0\}$:}
Let $x \in \As \cap \Bs$ (where the intersection is a  $C^*$-subalgebra of $\Cs$). 
Then 
 \[  x = \ub{\Re x}_{y} + i \ub{\Im x}_{z} \]
where $y,z \in (\As \cap \Bs)_{\text{sa}}$.
Then we have $y^2 =\ub{y}_{\in \As} \ub{y}_{\in \Bs} \in \As\Bs = \{0\}$, 
so $y=0$ and similarly $z = 0$, so $x =0$.


This means that $\Cs$ is the internal direct sum of $\As$ and $\Bs$ (as defined in (a)), 
so (a) gives that $\As$ and $\Bs$ are (closed) ideal of $\Cs$ and $\Cs\simeq \As \oplus \Bs$

\subsection*{c)}
Let $\As$ be a unital $C^*$-algebra with unit $e$. Let $\tAs$ be a defined as a $*$-algebra
(as we did when $\As$ is non-unital). Set $\As' = \As\times \{0\}$ and $\Bs = \C(1-e) =
\{\la (-e, 1): \la \in \C\}$. $\tAs$ is the internal direct sum of $\As'$ and 
$\Bs$ (as $*$-algebra): 

$\tAs = \As' + \Bs$: Let $(A, \la ) \in \tAs$. Then $(A,\la) = \ub{(A + \la e, 0)}_{\in \As'} 
+ \la \ub{(-e,1)}_{\in \Bs}$.

$\As'\cap \Bs = \{(0,0)\}$: Assume $(A,0) = \la (-e,1)$ for $A \in \As$, 
$\la \in \C$. Then $\la = 0$, so $A =0$.

$\As'\Bs = \{(0,0)\}$: $(A, 0)(\la(-e,1)) = \la(-Ae+A,0) = \la (0,0)$ for all 
$A \in \As$, $\la \in \C$.
Using $(a)$ we get that $\tAs = \As'\oplus \Bs$, but $\As'$ is $*$-isomorphic to 
$\As$ (via $(A,0)\mapsto A$) and $\Bs$ is $*$-isomorphic
to $\C$ (via $(\la(1-e) \mapsto \la)$.

It follows that $\tAs \simeq \As \oplus \C$
(as $*$-algebras). This implies that $\tAs$ can be given a $C^*$-algebra norm, i.e.
organized as a $C^*$-algebra. 

\section*{Definitions}
$\As$ is a $C^{*}$-algebra, $\vphi$ is a linear functional on $\As$. We call $\vphi$
\emph{faithful} if 
\[ \{A \in \As : \vphi(A^*A) = 0\} = \{0\} \] 

\section*{Ex 14}
Set $\As = C(\Om)$, where $\Om$ is a compact Hausdorff space. Let $\mu$
be a finite regular (means positive) Borel measure on $\Om$.
\[\vphi_{\mu} \colon \As \to \C, \qquad \vphi_{\mu} = \int_{\Om}f\d\mu \]
Let $S_{\mu} \coloneqq  \supp(\mu) = \{\om\in \Om : \mu(U)>0 \text{ for all open neighborhoods $U$ of $\om$}\}$.

\ul{$S_{\mu}$ is closed in $\Om$:} Assume $\om' \in \Om\setminus S_{\mu}$, 
then there exits a open neighborhood $V$ of $\om'$ such that $\mu(V)=0$.
Consider $\sigma \in V$. Then $V$ is an open neighborhood of $\sigma$ such that 
$\mu(V) = 0$, so $\sigma\not\in S_{\mu}$. Hence $V\subseteq\Om\setminus S_{\mu}$. 
Thus $\Om\setminus S_{\mu}$ is open. 

\ul{$\vphi_{\mu}$ is faithful if and only if $S_{\mu}= \Om$:}
Note first that $\mu(\Om\setminus S_{\mu}) = 0$. Indeed, consider $K\subseteq \Om\setminus S_{\mu}$, $K$ compact. By compactness we can find open sets $V_{1},\ldots,V_{n}$ such that 
$K = V_{1} \cup \cdots \cup V_{n}$ and $\mu(V_{j})= 0$ for $j=1,\ldots, n$.
Then $\mu(K) \leq \sum_{j=1}^{n}\mu(V_j) = 0$. So $\mu(K) = 0$. Hence 
\[\mu(\Om\setminus S_{\mu}) 
= \sup\{\mu(K): K \text{ compact }, K \subseteq \Om \setminus S_{\mu} \} = 0.
\]
\ul{Assume $S_{\mu}\neq \Om$:}
Let $\sigma \in \Om\smin S_{\mu}$. By Uryshons lemma we can pick $f(\sigma)=1$
and $f=0$  on $S_{\mu}$, so that $f\neq 0$ and 
\begin{align*}
    \vphi_{\mu}(f^*f) = \int_{\Om}|f|^2\d \mu = \ub{\int_{\Om\smin S_{\mu}} |f|^2\d\mu}_{\text{$=0$, since } \mu(\Om\smin S_{\mu})=0}
+\ub{\int_{S_{\mu}} |f|^2 \d \mu}_{=0,\text{ since } f=0} = 0  
\end{align*}
so $\vphi_{\mu}$ is not faithful. 

Conversely assume $S_{\mu} = \Om$.
Let $f\neq 0$, $f \in \As$. Then 
$M = ||f||_{\infty} > 0$. Then $U = |f|^{-1}((M/2,\infty)$ is an open set in $\Om$, which is non-empty, so $\mu(U)> 0$ (since $S_{\mu} = \Om$) 
and 
\[ \vphi(f^* f) = \int_{\Om} |f|^2 \d \mu \geq \int_{U} |f|^2 \d \mu \geq \frac{M^2}{2}\mu(U)> 0.\] 
This shows that 
$\vphi(f^*f) = 0 \implies f=0$ i.e. $\vphi_{\mu} $ is faithful.  

\section*{Ex 15}
$H$ Hilbert space. $\As$ is $C^*$-subalgebra of $\Bs(H)$, $\xi \in H$. Then 
\[ \vphi_{\xi} (A) = \ind{A\xi,\xi}, \quad A \in \As \]
gives a positive linear functional on $\As$.

\subsection*{a)}
$\vphi_{\xi}$ is faithful if and only if $\{A \in \As: A\xi=0\}= \{0\}$, i.e. $\xi$
is separating for $\As$.

Indeed, $\vphi_{\xi}(A^*A)= \ind{A\xi,A\xi} = ||A\xi||^2$ for all $A \in \As$, so
if $\vphi_{\xi}(A^*A) = 0 \iff A\xi =0$ and the assertion is obvious. 

\subsection*{b)}
$H = \C^3$, $\As = M_2(\C)\oplus \C = \{[a_{ij}]\in \C^{3\times 3} :
a_{13}=a_{23} = a_{31} = a_{32} = 0\}$.
Let $\xi \in \C^3$. \ul{Then $\vphi_{\xi}$ is not faithful:}
Note that $\{A \in \As : A\xi =0\} = \ker T_{\xi}$ where $T_{\xi}\colon\As\to\C^3$
is given by $T_{\xi}(A)= A\xi$. Since \[\dim \ker T_{\xi} = \ub{\dim \As}_{5}  -
\ub{\dim \text{Range}T_{\xi}}_{\leq 3} \geq 2.\]
We then get that 
$\{A\in \As: A\xi = 0 \}\neq\{0\}$ so $\vphi_{\xi}$ is not faithful. 

\subsection*{c)}
$\xi = [1,0,1]^T$, $\xi_2 = [0,1,0]^T$. Set $\vphi = \vphi_{\xi_1} + \vphi_{\xi_2}$ 
which is positive on $\As$. 
\ul{Then $\vphi$ is faithful:}
Assume $\vphi(A^*A) = 0$ for $A \in \As$. Then we have 
\begin{align*}
   \vphi_{\xi_1}(A^*A) + \vphi_{\xi_2}(A^*A) = 0 \\ 
   ||A\xi_1||^2 + ||A\xi_2||^2 = 0 \\
   A\xi_1 = A\xi_2 = 0 
\end{align*}
but 
\begin{align*}
    A\xi_1 &= 0 &&\implies a_{11} = a_{21} = a_{33} = 0 \\ 
    A\xi_2 &= 0 &&\implies a_{12} = a_{22} = 0 \\ 
\end{align*}
so we get that $A = 0$.

\subsection*{c)}
Let $t_1, t_2 > 0$. The 
$\psi = t_1\vphi_{\xi_1} + t_2\vphi_{\xi_2}$ is a positive linear functional on 
$\As$. We have that 
\begin{align*}
||\psi|| &= ||t_1\vphi_{\xi_1}||+||t_2\vphi_{\xi_2}|| \\
&= t_1\vphi_{\xi_1}(I) + t_2\vphi_{\xi_2}(I) = 2t_1 + t_2
\end{align*}
since $||\xi_1||^2 = 2$ and $||\xi_2||^2 = 1$.
Hence $\psi$ is a state on $\As$ when 
$2t_1 + t_2 = 1$ e.g. $t_1 = t_2 = 1/3$ or $t_1 = 1/2$, $t_2 = 1/2$, in which case 
it is faithful (by similar argument as above).

\section*{Ex 16}
A linear functional $\tau$ is called \emph{tracial} if $\tau(AB)= \tau(BA)$ for
all $A,B \in \As$. Let
$\As$ be a \ul{unital} $C^*$-algebra and $\tau$ is a linear functional on $\As$. 

$\tau$ is tracial if and only if $\tau(UAU^*) = \tau(A)$ for all $A \in \As$,
$U \in \Us(\As)$.

($\implies$) If $\tau$ is tracial, then $\tau(UAU^*) = \tau(AU^*U)= \tau(A)$.

($\impliedby$) Assume RHS holds. To show that the LHS holds it suffices to 
show that we have $\tau(AU) = \tau(UA)$ for all $A \in \As$ and $U \in \Us(\As)$,
because any $B\in \As$ is a linear combination of unitaries in $\As$.
But we have $\tau(UA) = \tau(UAUU^*) = \tau(AU)$ for all $A \in \As$ and $U\in \Us(\As)$,
where we in the last equality used 
the RHS. 

\section*{Ex 17}
Let $\As = M_{n}(\C)$ and 
$\Tr\colon \As\to\C$ where $\Tr$ is defined by 
$\Tr(A) = \sum_{i=1}^{n} a_{ii}$ for $A = [a_{i,j}]_{i,j=1}^{n}$.
\subsection*{a)}
$\Tr$ is a faithful tracial positive linear functional: Linearity is obvious.
Note that $\Tr(A)=\sum_{j=1}^{n}\ind{Ae_j,e_j}$ where $e_1,\ldots,e_n$ is the 
standard basis on $C^n$. Hence 
$\Tr(A^*A) = \sum_{j=1}^{n} ||Ae_j||^2 = \sum_{i,j=1}^{n} |a_{i,j}|^2 \geq 0$
and $\Tr(A^*A) = 0 \iff a_{ij}=0 \text{ for all } i,j\in\{1,\ldots,n\} \iff A  = 0$.
So $\Tr(A^*A) = 0$ is positive and faithful. 

Finally, 
\begin{align*}
\Tr(AB)&=\sum_{j=1}^{n} (AB)_{jj} = \sum_{j=1}^{n}\sum_{i=1}^{n}a_{ji}b_{ij} \\&=
\sum_{i=1}^{n}\sum_{j=1}^{n}b_{ij}a_{ji}  = \sum_{i=1}^{n}(BA)_{ii} = \Tr(BA)
\end{align*}
for all $A,B \in \As$.


\subsection*{b)}
Let $\{\xi_1,\ldots, \xi_n\}$ be an orthonormal basis for $\C^n$. Let 
$U = [\xi_1,\ldots, \xi_n] \in \Us(\As)$. Then $Ue_j = \xi_j$ for all 
$j=1,\ldots,n$. Then we get 
\begin{align*}
    \sum_{j=1}^{n} \ind{A\xi_j,\xi_j} = \sum_{j=1}^{n} \ind{AUe_j,Ue_j}
    =\sum_{j=1}^{n} \ind{U^*AUe_j,e_j} = \Tr(U^*AU) = \Tr(VAV^*) = \Tr(A)
\end{align*}
for all $A \in \As$, where $V = U^*$. So $\Tr(A) =\sum_{j=1}^{n} \ind{A\xi_j,\xi_j}$.

\subsection*{c)}
$S \in \As$. Define $\vphi_{S}\colon \As \to \C$ by $\vphi_{S}(A) = \Tr(AS)$.
Then $\vphi_{S}$ is clearly linear, hence $\vphi_{S}$ is bounded since $\As$
is finite dimensional. Then map $S\mapsto\vphi_S$ from $\As$ into $\As^*$
is also linear (easy). 

Moreover it is injective:
Assume $\vphi_S = 0$, so $\Tr(SA) = 0$ for all $A \in \As$. But then 
$\Tr(S^*S) = \Tr(SS^*) = 0$, but $\Tr$ is faithful, so $S = 0$. 

We know that $\dim \As = \dim \As^*$, so we can conclude that $S \mapsto \vphi_{S}$
is bijective. Hence this gives a vector space isomorphism between $\As$ and 
$\As^*$. 

We have $S \geq 0$ in $\As$ if and only if $\vphi_S\geq 0$ in $\As^*$:
Assume $S\geq 0$. Then 
\begin{align*}
\vphi_S (A^*A) &= \Tr(SA^*A) = \Tr(S^{1/2}S^{1/2}A^*A)\\
&= \Tr(S^{1/2}A^*AS^{1/2})= \Tr((AS^{1/2})^*AS^{1/2})\geq 0.
\end{align*}
Conversely, assume $\vphi_s\geq 0$. We first show that $S$ is self-adjoint. We 
know that $\vphi_S$ is self-adjoint, since $\vphi_S\geq 0$ i.e.
\[\ol{\Tr(SA)} = \Tr(SA^*) = \Tr(A^*S) \quad\text{for all }A\in \As.\]
Hence $\Tr((S-S^*)A) = \Tr(SA)- \Tr(S^*A) = \Tr(SA)- \ol{\Tr(A^*S)} =\Tr(SA)- \Tr(SA) = 0$
for all $A \in \As$. 
In particular, we set 
$\Tr((S-S^*)^*(S-S^*)) = \Tr((S-S^*)(S-S^*)^*) =0$, so $S-S^*=0$ since 
$\Tr$ is faithful. 

Now show that $S\geq 0$. It suffices to show all its eigenvalues are non-negative.
Pick an orthonormal basis $\{\xi_1,\ldots,\xi_n\}$ of eigenvectors of $S$ i.e.
$S\xi_j = \la_j\xi_j$, $j=1,\ldots,n$.
Then we have $S = \sum_{j=1}^{n} \la P_{\xi_j}$ where 
$P_{\xi_j}=$ orthonormal projection from $C^n$ onto $\Span{\xi_j}$.
Now each $P_{\xi_j}\geq 0$ for $j=1,\ldots,n$, so 
$\vphi_{S}(P_{\xi_j})\geq 0$ for $j=1,\ldots,n$.
We get 
\begin{align*}
    \la_j = \la_j \Tr(P_{\xi_j}) =\Tr(\la_j P_{\xi_j})
    = \Tr(SP_{\xi_j}) = \vphi_S(P_{\xi_j})\geq 0  
\end{align*}
for $j=1,\ldots,n$.
Where we used that 
\begin{align*}
    \Tr(P_{\xi_j})= \sum_{i=1}^{n} \ind{P_{\xi_j}\xi_i,\xi_i} = \ind{\xi_j,\xi_j} = 1
\end{align*}
Notice also that $SP_{\xi_j} = \(\sum_{j=1}^{n}\la_jP_{\xi_j}\)P_{\xi_i} = \la_i P_{\xi_i}$.
Hence $S\geq 0$ as desired.
Finally in this case we get that 
$||\vphi_S|| = \vphi_s(I) = \Tr(S)$. 

\subsection*{d)}
$\vphi = t\Tr$ for some $t \geq 0$:
By $c)$ we can write $\vphi = \vphi_{S}$ for some $S \in \As^+$. Since
$\vphi$ is tracial, we get 
\begin{align*}
\vphi_S(AB) &= \vphi_{S}(BA)\quad\text{ for all }A,B\in \As\text{ which implies}\\ 
\Tr(SAB) &= \Tr(SBA) \quad\text{ for all }A,B\in \As. 
\end{align*}
so $\Tr(B(SA-AS))=0$ for all $A,B\in \As$, so 
$\Tr((SA-AS)^*(SA-AS)) = 0$ for all $A \in \As$.
Hence $SA=AS$ for all $A \in \As$.

Let $E_{kl}\in \C^{n\times n}$ be given by 
\begin{align*}
    (E_{kl})_{ij} = \begin{cases} 1&\text{if }i=k \text{ and }j=l \\
    0 &\text{otherwise} \end{cases}
\end{align*}
Then 
\[
SE_{kl} = \begin{bmatrix} 
0&\ldots &\overset{l}{s_{k1}} &\ldots &0\\
0&\ldots &s_{k2} &\ldots &0\\
\vdots&\ddots &\vdots &\ddots &\vdots\\
0&\ldots &s_{kn} &\ldots &0\\
  \end{bmatrix}
\text{ and }
E_{kl}S = \begin{bmatrix} 
0&\ldots,&\ldots & 0\\
\vdots&\ddots,&\ddots & \vdots\\
s_{1l} & s_{2l}&\ldots & s_{nl} \\
\vdots&\ddots,&\ddots & \vdots\\
0&\ldots,&\ldots & 0\\
\end{bmatrix}
\]
so $SA=AS$ implies that $SE_{kl}=E_{kl}S$ for all $k,l\in \{1,\ldots,n\}$.
this implies that $S_{kk}= S_{ll}$ and for all $k,l\in \{1,\ldots,n\}$
and $S_{kl} = 0$ for $k\neq l$. So
\[S = \begin{bmatrix} 
t & 0 &\ldots & 0 \\
0 & t &\ldots & 0 \\
\vdots & \ddots &\ddots & \vdots \\
0 & 0 &\ldots & t \\
 \end{bmatrix} = tI\]
and $t \geq 0$ since $S \geq 0$, so $\vphi = \vphi_s = t\Tr$.
Note that $||\vphi|| = t \Tr(I) = t n$, so $\vphi$ is a state when 
$t=1/n$. Hence the only tracial state on $\As$ is $\tfrac{1}{n}\Tr$ (normalized trace).


\section*{Ex 18}
Let $\As$ is a $C^*$-algebra, $\vphi \in \As^*$. Define $\vphi^*(A) = \ol{\vphi(A^*)}$
for all $A \in \As$. 
\subsection*{a)}
$\vphi^*$ is clearly linear. Moreover,
$|\vphi^*(A)| = |\vphi(A^*)| \leq ||\vphi||\,||A^*|| =||\vphi||\,||A||$
for all $A \in \As$.
Thus $\vphi^*$ is bounded, so $\vphi^* \in \As^*$ with $||\vphi^*||\leq||\vphi||$.
Now $(\vphi^*)^* (A) = \ol{\vphi^*(A^*)} = \vphi((A^*)^*) = \vphi(A)$ for all
$A \in \As$, so $(\vphi^*)^* = \vphi$ (involution). 
This gives $||\vphi||=||(\vphi^*)^*|| \leq ||\vphi^*||$. All together we
get $||\vphi^*|| = ||\vphi||$.
\subsection*{b)}
Strait forward, use the definition. 

\section*{Ex 20}
Let $\{H_j\}_{j \in J}$ be a family of Hilbert spaces. Set
\[  H = \{ \{\xi_j\} \in \prod_{j\in J} H_j : \sum_{j\in J} ||\xi_j||^2 < \infty  \} \]
\subsection*{a)}
Let $\xi = \{\xi_j\}, \eta= \{\eta_j \} \in H$. Then $\sum_{j\in J}
\ind{\xi_j,\eta_j}$ is convergent. Enough to show absolute convergence. 
We have 
\begin{align*}
    \sum_{j\in J} |\ind{\xi_j,\eta_j}| \underset{\text{C.S. for each }j}{\leq}
     \sum_{j\in J} ||\xi_j||\eta_j||
    \underset{\text{Holder}}{\leq} \( \sum_{j\in J} ||\xi_j||^2 \)^{1/2}
\( \sum_{j\in J} ||\eta_j||^2 \)^{1/2}
\end{align*}
where we used the holder inequality in $\ell^2(J)$ using that 
$\sum_{j\in J} ||\xi_j||^2, \sum_{j\in J} ||\eta_j||^2< \infty$.  

\subsection*{b)}
It is now easy to check that $(H, \ind{\cdot,\cdot})$, becomes a Hilbert space 
which we denote $\oplus_{j\in J} H_j$ and call the direct sum of the family
$\{H_j\}$.

\subsection*{c)}
For each $j\in J$, let $T_j \in \Bs(H_j)$. 
Assume that $M\coloneqq\sup_{j\in J} ||T_j|| < \infty$.
For $\xi  =\{\xi_j\} \in H$,
define $T\xi = \{T_j\xi_j\}_{j\in J} \in \prod_{j\in J} H_j$.
Then $T\xi \in H$, since 
$\sum_{j\in J}||T_j \xi_j||^2 \leq \sum_{j\in J} ||T_j||^2 ||\xi_j||^2 \leq M^2
\sum_{j\in J}  ||\xi_j||^2$, so we get a map 
$T \colon H \to H$ given by $\xi \mapsto T\xi$ which is as usual linear. 
Moreover, we have 
 \begin{align*}
     ||T\xi||^2  = \sum_{j\in J} ||T\xi_j||^2 \leq M^2 ||\xi||^2
 \end{align*}
so $T$ is bounded with $||T||\leq M$.
Consider $\xi_k \in H_k$, $||\xi_k||\leq 1$ for some $k \in J$.
Define $\tilde{\xi} = \{\tilde{\xi}_j\}$ where 
$\tilde{\xi}_j = \xi_k$ where $j=k$ and $\tilde{\xi}_j = 0$
otherwise.
Then $||\tilde{\xi}|| = ||\xi_k|| \leq 1$ and 
$T\tilde{\xi} = \{\eta_j\}$, where 
\[\eta_j = \begin{cases} T_k\xi_k & j=k \\ 0 &\text{otherwise} \end{cases}\]
so $||T(\tilde{\xi}) || \leq ||T_k\xi_k|| \leq ||T|| \,||\tilde{\xi}|| \leq ||T||$.
so we get $||T_k||\leq ||T||$.
Taking the sup of all $k$'s in $J$ we get $M\leq ||T||$. Hence $\sup_{j\in
J}||T_j|| = ||T||$.

\subsection*{d)}
We will define $T$ by $\tilde{\oplus}_{j\in J}T_j$. 
so
\[ \(\top_{j\in J} T_j\) \( \{\xi_j\}_{j\in J}\) = \{T_j\xi_j\}_{j\in J}\]
Assume that each $\As_j$ is a $C^*$-subalgebra of $\Bs(H_j)$ and form
$\As \coloneqq \oplus_{j\in J}\As_j$.
Define $\pi\colon \As\to\Bs(H)$ by
\[ \pi (\{T_j\}) = \top_{j\in J} T_j,\quad\text{ for each } \{T_j\} \in \As\]
Then $\pi$ is an \ul{isometric $*$-homomorphism}.
Let us show that $\pi$ is $*$-preserving, that is 
$\pi(\{T_j\}^*) = \pi(\{T_j\})^*$. Notice that 
$\pi(\{T_j\}^*) = \pi(\{T_j^*\}) = \top_{j\in J} T_{j}^{*}$ and that 
$\pi(\{T_j\})^* = (\top_{j\in J}T_j)^*$. Let 
$\xi = \{\xi_j\}, \eta= \{\eta_j \} \in \As$. Then 
\begin{align*}
    \ind{\top_{j\in J} T_{j}^{*}\xi,\eta} 
    &= \ind{\{T_{j}^{*}\xi_j\},\{\eta_j\}}  
    = \sum_{j\in J} \ind{T_{j}^{*}\xi_j,\eta_j} \\
    &= \sum_{j\in J} \ind{\xi_j,T_{j}\eta_j} 
    = \ind{\{\xi_j\},\{ T_j\eta_j\}} = \ind{\xi,(\top_{j\in J}T_j)\eta}
\end{align*}
Thus $(\top_{j\in J}T_j)^* = \top_{j\in J}T_j$ as desired. 
Note that $\pi$ is isometric since 
$||\pi((\{T_j\})|| = ||\top_{j\in J} T_j|| \underset{\text{part c}}{=} \sup_{j\in J} 
||T_j|| = ||\{T_j\}||$.







































\end{document}


